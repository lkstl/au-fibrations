\documentclass[a4paper]{amsart}

\author{Ingo Blechschmidt}
\author{Matthias Hutzler}
\author{Lukas Stoll}
\title{Fibered categories over arithmetic universes}
\date{\today}

\usepackage{mathtools}
\usepackage{fontspec}
\usepackage{unicode-math}
\usepackage[protrusion=true,expansion=true]{microtype}
\usepackage[style=alphabetic,useprefix=true]{biblatex}

\theoremstyle{plain}
\newtheorem{theorem}{Theorem}[section]
\newtheorem*{theorem*}{Theorem}
\newtheorem{proposition}[theorem]{Proposition}
\newtheorem*{proposition*}{Proposition}
\newtheorem{lemma}[theorem]{Lemma}
\newtheorem*{lemma*}{Lemma}
\newtheorem{corollary}[theorem]{Corollary}
\newtheorem{corollary*}{Corollary}

\theoremstyle{definition}
\newtheorem{definition}[theorem]{Definition}
\newtheorem*{definition*}{Definition}
\newtheorem{example}[theorem]{Example}
\newtheorem*{example*}{Example}
\newtheorem{examples}[theorem]{Examples}
\newtheorem*{examples*}{Examples}
\newtheorem{remark}[theorem]{Remark}
\newtheorem*{remark*}{Remark}

\addbibresource{Streicher2018.bib}
\addbibresource{Maietti2010.bib}

\begin{document}
\maketitle

\section{Arithmetic Universes}

\begin{definition}
  \emph{disjoint coproduct}
\end{definition}

\begin{definition}
  \emph{stable colimit}
\end{definition}

\begin{lemma}
  In a category with stable coproducts, pullbacks of disjoint coproducts are disjoint.
\end{lemma}

\begin{definition}
  \emph{equivalence relation},
  \emph{effective quotient}
\end{definition}

\begin{definition}
  \emph{natural numbers object},
  \emph{list object},
  \emph{parametrized list object}
\end{definition}

\begin{definition}
  A \emph{list-arithmetic universe} is a category $𝒞$ which has …
  \begin{enumerate}
    \item … finite limits,
    \item … stable finite disjoint coproducts,
    \item … stable effective quotients of monic equivalence relations,
    \item … parametrized list objects.
  \end{enumerate}
\end{definition}

\section{Fibered categories}

\begin{definition}
  Let $P:ℰ → ℬ$ be a functor.
  \begin{enumerate}
    \item A morphism in $ℰ$ is \emph{vertical} if its image under $P$ is an identity morphism.
      \marginpar{Automorphism?}
    \item A morphism $φ$ in $ℰ$ is \emph{cartesian} if every morphism $ϑ$ with the same codomain factors through $φ$, provided this holds for their images under $P$.
      Moreover, the factorization through $φ$ is required to be unique.
      \marginpar{Diagram!}
    \item A \emph{cartesian lift through $P$} of a morphism $u$ in $ℬ$ is a cartesian morphism in $ℰ$ which is mapped to $u$ by $P$.
    \item The functor $P$ is a \emph{fibration} if every morphism $u$ in $ℬ$ admits a cartesian lift through $P$.
      In this case, we also say that $ℰ$ is \emph{fibered over $ℬ$ by $P$}.
  \end{enumerate}
\end{definition}

\begin{definition}
  Let $P:ℰ → ℬ$, $Q:ℱ → ℬ$ be fibered categories.
  \begin{enumerate}
    \item A \emph{fibered functor} ...
    \item A \emph{vertical natural transformation} ...
    \item A \emph{fibered adjunction} ...
  \end{enumerate}
\end{definition}

\begin{examples}
  \hfill
  \begin{enumerate}
    \item \emph{codomain fibration}
    \item \emph{Grothendieck contruction}
  \end{enumerate}
\end{examples}

\nocite{*}
\printbibliography
\end{document}
